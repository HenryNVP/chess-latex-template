\documentclass[a4paper,12pt]{book}

\title{Chess Latex Template}
\author{HenryNVP}

\usepackage[inner=2.5cm,outer=2.5cm,top=2.5cm,bottom=2.5cm,headheight=7mm,headsep=1mm,heightrounded]{geometry}
\usepackage{fancyhdr}
\pagestyle{fancy}

% Chess package
\usepackage{chessboard}
\usepackage{xskak}

\begin{document}
	
\newcommand\textboard[2][]{
	\noindent
	\begin{minipage}[t]{0.49\textwidth}
		\centering\chessboard[setfen=#2]\par
		\raggedright#1\strut
	\end{minipage}
	\hfill\ignorespaces}

\newcounter{diagrams}
\newcommand\printcounts{
	\normalsize\refstepcounter{diagrams}\thediagrams}
	
\newcommand\tacticboard[2][]{
	\noindent
	\begin{minipage}[t]{0.49\textwidth}
		\centering\printcounts\chessboard[setfen=#2]\par
		\Large#1\strut
	\end{minipage}
	\hfill\ignorespaces}
	
\setchessboard{showmover=true, moversize=24px, moverstyle=circle}
	
% --------Theory--------
\chapter*{Double attack}

The double attack happens when a single move by a player creates two simultaneous threats against their opponent.

\newchessgame
\textboard[\mainline{1.Ne5} attacks two pieces with higher values ]{r5k1/p2q1ppp/2r5/2np4/2N5/5Q1P/PP3PP1/R4RK1 w - - 0 1}
\newchessgame
\textboard[\mainline{1.Qc8+ Kh7 2.Qxg4}]{6k1/5pp1/7p/3q4/6r1/2Q5/5PPP/5RK1 w - - 0 1}
\newchessgame
\textboard[\mainline{1.Qc5} threatens checkmate and attacks unprotected pieces]{7k/pb4np/p5p1/6b1/Pr3N2/4Q1P1/1P5P/5BKR w K - 0 1}

% -------Exercise-------
\lhead{\Large\bfseries Double Attack Exercises}

\tacticboard[\dotfill]{3r2k1/p1q2pp1/1p5p/2p2Q2/8/1P4PP/P4PB1/2B3K1 b - - 0 1}
\tacticboard[\dotfill]{2r5/p5kp/1p3pp1/8/2B5/2P3Pn/1P1R1P1P/7K w - - 0 1}
\vspace{\fill}
\tacticboard[\dotfill]{2k1r3/pp3ppp/2pp4/Q1b5/3n1B2/P7/1PP2PPP/2K5 b - - 0 1}
\tacticboard[\dotfill]{1r3k2/bp5p/p4pp1/4p3/2P1P2n/1P3P1P/P2Q2P1/7K w - - 0 1}
\vspace{\fill}
\tacticboard[\dotfill]{5b2/1bp2ppk/p5qp/3P4/2P5/5P2/1B1NQ1PP/6K1 b - - 0 1}
\tacticboard[\dotfill]{3qr1k1/1p3pbn/p2p2p1/3P2Pp/5B1P/5Q2/PPPR4/1K4R1 b - - 0 1}
\vspace{\fill}
	
\end{document}


